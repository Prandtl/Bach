\documentclass[12pt]{report}
\usepackage[russian]{babel}
\usepackage[utf8]{inputenc}
%\usepackage[a4paper,margin=15mm, lmargin=30mm]{geometry}
\usepackage{indentfirst}
%\usepackage{calc}
%\usepackage{setspace}
\usepackage{amsmath,amssymb,amsfonts}
\usepackage{amsthm}
\usepackage{cite}
\usepackage[pdftex]{graphicx}

\textheight 24cm
\voffset -2cm
\textwidth 17cm
\hoffset -1.5cm
\topmargin 0mm

\renewcommand{\bottomfraction}{1.0}
  % Какая часть снизу листа может быть занята картинками
\renewcommand{\textfraction}{0.0}
  % Какая часть листа должна быть занята текстом

\newtheorem{theorem}{Теорема}
\newtheorem{definition}{Определение}
\newtheorem{ex}{Пример}
\newtheorem{tab}{Таблица}
\newtheorem{sat}{Утверждение}


\begin{document}
\pagestyle{empty} % нумерация выкл.
\begin{center}
\small
МИНИСТЕРСТВО ОБРАЗОВАНИЯ И НАУКИ РОССИЙСКОЙ ФЕДЕРАЦИИ\\
Федеральное государственное автономное образовательное учреждение высшего образования\\
УРАЛЬСКИЙ ФЕДЕРАЛЬНЫЙ УНИВЕРСИТЕТ \\
имени первого Президента России Б.Н. Ельцина\\
\vspace{1em}
ИНСТИТУТ ЕСТЕСТВЕННЫХ НАУК И МАТЕМАТИКИ\\
Кафедра математической экономики (TODO: Место выполнения ВКР и работы научного руководителя (консультанта) ВКР (либо кафедра, либо департамент))
\end{center}

\vspace{1em}

\begin{center}

\large
АДАПТАЦИЯ АЛГОРИТМА ПАРАЛЛЕЛЬНОГО СТОХАСТИЧЕСКОГО ГРАДИЕНТНОГО СПУСКА К ЗАДАЧАМ МЕХАНИКИ СПЛОШНОЙ СРЕДЫ
\vspace{1em}

\normalsize
Направление подготовки 09.03.03 <<Прикладная информатика>>\\
\vspace{1em}
Образовательная программа <<TODO: ???>>
\end{center}
\vspace{1em}

\begin{tabular}[t]{@{}l}
  Допустить к защите:\\
  Директор департамента:\\ {todo: title}М. О. Асанов\\
  \underline{\hspace{5cm}}\\
  \vspace{1em}\\
  Нормоконтроллер:\\ {todo: find that person's name}\\
  \underline{\hspace{5cm}}
\end{tabular}
\hfill% move it to the right
\begin{tabular}[t]{l@{}}
    Выпускная квалификационная\\работа бакалавра\\
    \textbf{Учанева}\\
    \textbf{Василия Вячеславовича}\\
    \underline{\hspace{5cm}}\\
    \vspace{1em}\\
    Научный руководитель:\\ к.ф.-м.н. В. С. Зверев\\
    \underline{\hspace{5cm}}\\
\end{tabular}
%
% \begin{flushright}
%   Оценка работы\underline{\hspace{4cm}}
%
%   Руководитель от УрФУ\underline{\hspace{4cm}}
% \end{flushright}


%
% \vspace{200pt}
% Руководитель	\underline{\hspace{8,6cm} TODO: имя}
%
% Студент \underline{\hspace{9,1cm}	Учанев Василий Вячеславович}
%
%
% \small Специальность (направление подготовки) \underline{\hspace{1cm} прикладная информатика}
%
% Группа МЕН-430107
% \vspace{33pt}
% \normalsize

\vspace*{\fill}
\begin{center}
Екатеринбург 2017 г.
\end{center}

\newpage
\pagestyle{plain} % нумерация вкл.
\chapter*{Реферат}
Егармин А.В. - -//- -: стр. - -,

Ключевые слова: АЛГОРИТМЫ КОРРЕКЦИИ, РОБОТ, ПЛАТФОРМА, КОРРЕКЦИЯ ДВИЖЕНИЯ, ПИД-РЕГУЛЯТОР.

В данной работе рассмотрено два подхода коррекции движнения автономной платформы при помощи ПИД-регуляторов. Первый --- упрощенная модель движения, вдоль прямой, сонаправленной с текущим направлением платформы. Второй --- модель движения, в которой коррекция движения платформы производится к произвольной прямой или ломанной траектории.

\newpage
%\pagestyle{plain} % нумерация вкл.
\tableofcontents
%\setstretch{1.0}

\chapter{Введение}
Для участия в робототехнических соревнования Eurobot-2017 было необходимо разработать автономного робота (роботов), выполняющего некоторый набор действий на игровом столе. Для выполнения задач требовалось разработать алогоритм передвижения автономной платформы из заданной зоны в набор точек последовательно. При таком передвижении важно, чтобы перемещение было как можно более точным, а также удовлетворяло таким параметрам, как отклонение за время движения и требуемое время на выполнение передвижения.

Для решения данной задачи было создано два алгоритма коррекции, основанные на пид-регуляторах. Первый из них предлагает движение вдоль прямой, сонаправленной с текущим направлением платформы, разворот на месте на заданный угол и движение до некоторой точки по плоскости стола. Второй алгоритм позволяет корректироваться к произвольной прямой (или ломанной линии) на плоскости, в том числе к прямым, не сонаправленным с направлением платформы и к прямым, лежащим далеко от текущего положения робота. Кроме того, данный алгоритм можно усовершенствовать для движения по некоторому классу кривых (первая производная кривой непрерывна и лежит в пространстве игрового стола).

В ходе реализации проекта был создан программный комплекс, симулирующий движение платформы по плоскости и наглядно демонстрирующий коррекцию движения вдоль ломанной траектории.

Кроме того, первый алгоритм реализован на двух различных роботах и использовался в вышеуказанных соревнованиях. Второй алгоритм был частично протестирован на одной из платформ, показал неудовлетворительные результаты в связи с возникшими транспортными расходами и не был реализован на основной платформе в связи с ограниченным сроком разработки и своей сложностью.

\chapter{Постановка задачи}
\end{document}
