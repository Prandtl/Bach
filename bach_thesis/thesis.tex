\documentclass[12pt]{report}
\usepackage[utf8]{inputenc}
\usepackage[russian]{babel}
%\usepackage[a4paper,margin=15mm, lmargin=30mm]{geometry}
\usepackage{indentfirst}
%\usepackage{calc}
%\usepackage{setspace}
\usepackage{amsmath,amssymb,amsfonts}
\usepackage{amsthm}
\usepackage{cite}
\usepackage[pdftex]{graphicx}

\textheight 24cm
\voffset -2cm
\textwidth 17cm
\hoffset -1.5cm
\topmargin 0mm

\renewcommand{\bottomfraction}{1.0}
  % Какая часть снизу листа может быть занята картинками
\renewcommand{\textfraction}{0.0}
  % Какая часть листа должна быть занята текстом

\newtheorem{theorem}{Теорема}
\newtheorem{definition}{Определение}
\newtheorem{ex}{Пример}
\newtheorem{tab}{Таблица}
\newtheorem{sat}{Утверждение}


\begin{document}
\pagestyle{empty} % нумерация выкл.
\begin{center}
\small\bf Министерство образования и науки Российской Федерации

ФГАОУ ВО <<УрФУ имени первого Президента России Б.Н. Ельцина>>

\rm Кафедра алгебры и дискретной математики
\end{center}

\normalsize
\vspace{2cm}

\begin{flushright}
  Оценка работы\underline{\hspace{4cm}}

  Руководитель от УрФУ\underline{\hspace{4cm}}
\end{flushright}

\vspace{3cm}
\begin{center}
Алгоритмы коррекции движения автономного робототехнического комплекса

\vspace{1cm}\normalsize
Выпускная квалификационная работа
\end{center}

\vspace{200pt}
Руководитель	\underline{\hspace{8,6cm}} Кумков Сергей Сергеевич

Студент \underline{\hspace{9,1cm}}	Егармин Алексей Валерьевич


\small Специальность (направление подготовки) математика

Группа МЕН-430111
\vspace{33pt}
\normalsize
\begin{center}
Екатеринбург 2017 г.
\end{center}

\newpage
\pagestyle{plain} % нумерация вкл.
\chapter*{Реферат}
Егармин А.В. - -//- -: стр. - -,

Ключевые слова: АЛГОРИТМЫ КОРРЕКЦИИ, РОБОТ, ПЛАТФОРМА, КОРРЕКЦИЯ ДВИЖЕНИЯ, ПИД-РЕГУЛЯТОР.

В данной работе рассмотрено два подхода коррекции движнения автономной платформы при помощи ПИД-регуляторов. Первый --- упрощенная модель движения, вдоль прямой, сонаправленной с текущим направлением платформы. Второй --- модель движения, в которой коррекция движения платформы производится к произвольной прямой или ломанной траектории.

\newpage
%\pagestyle{plain} % нумерация вкл.
\tableofcontents
%\setstretch{1.0}

\chapter{Введение}
Для участия в робототехнических соревнования Eurobot-2017 было необходимо разработать автономного робота (роботов), выполняющего некоторый набор действий на игровом столе. Для выполнения задач требовалось разработать алогоритм передвижения автономной платформы из заданной зоны в набор точек последовательно. При таком передвижении важно, чтобы перемещение было как можно более точным, а также удовлетворяло таким параметрам, как отклонение за время движения и требуемое время на выполнение передвижения.

Для решения данной задачи было создано два алгоритма коррекции, основанные на пид-регуляторах. Первый из них предлагает движение вдоль прямой, сонаправленной с текущим направлением платформы, разворот на месте на заданный угол и движение до некоторой точки по плоскости стола. Второй алгоритм позволяет корректироваться к произвольной прямой (или ломанной линии) на плоскости, в том числе к прямым, не сонаправленным с направлением платформы и к прямым, лежащим далеко от текущего положения робота. Кроме того, данный алгоритм можно усовершенствовать для движения по некоторому классу кривых (первая производная кривой непрерывна и лежит в пространстве игрового стола).

В ходе реализации проекта был создан программный комплекс, симулирующий движение платформы по плоскости и наглядно демонстрирующий коррекцию движения вдоль ломанной траектории.

Кроме того, первый алгоритм реализован на двух различных роботах и использовался в вышеуказанных соревнованиях. Второй алгоритм был частично протестирован на одной из платформ, показал неудовлетворительные результаты в связи с возникшими транспортными расходами и не был реализован на основной платформе в связи с ограниченным сроком разработки и своей сложностью.

\chapter{Постановка задачи}

\section{Механика автономной платформы}
При разработке автономных платформ использовалась классическая компановка с двумя колесами, приводимыми в движение двигателями постоянного тока, и опорой/опорами в виде поворотного колеса или шаровой опоры. Были разработы две платформы. Для основной (большой) платформы основые колеса были размещены на отдельных валах и связаны с двигателями ременными передачами. Для второстепенной (малой) платформы колеса размещены на валах двигателей.

Основная сложность при работе с двигателями постоянного тока заключается в том, что фактически можно управлять только напряжением на контактах двигателя, что хорошо связано с моментом силы на валу. Однако при использовании двигателей для передвижения необходимо контролировать скорость вращения.

В качестве обратной информации и источника данных для определения текущего положения используются датчики (энкодеры), установленные на двигателях. Конструкция основного робота такова, что проскальзывание ремней приводов исключено при нагрузках не превыщающих рассчетных (в частности, изготовлены натяжители ремней и использованы зубчатые ремни и шкивы). Кроме того, минимизированы люфты компонентов ходовой части. Для малой платформы, вышеописанныя схема не использовалась в силу малых габаритов платформы, а также всвязи с малой нагрузкой (малым весом платформы) на оси двигателей, что позволяло разместить колеса на валах двигателей.

На двигателях установлены редукторы с соотношением 1:26 и энкодеры, фиксирующие 64 положения на один оборот диска энкодера. Энкодеры закреплены на ось двигателя. Таким образом на один оборот вала редуктора, фиксируется $p = 1664$ положения. На вал редуктора закреплен, в зависимости от платформы, либо шкив привода, либо колесо.

На большой платформе использовались колеса радиусом  $R = 50$ мм, что позволяло измерять изменение положения колеса с точностью $A = \frac{2  \pi R }{p} \approx 0,19$ мм. На малой платформе использованы колеса радиусом \marginpar{указать радиус} $r = $ мм, что дает точность $a = \frac{2  \pi r }{p} \approx $ мм.

Ведущие колеса размещены ассиметрично геометрической форме платформ.


\section{Математическая модель двухколесной платформы}
Под положением платформы в пространстве будем понимать тройку $(x, y, \alpha)$, где $x, y$ --- координаты по осям базиса $Ox, Oy$, соответственно, $\alpha$ --- угол направления робота относительно оси $Ox$. При этом данная тройка характеризует точку, находящуюся в центре оси ведущих колес, при этом направлением платформы будем считать направление, сонаправленное с ходом движения колес.

ЗДЕСЬ ДОЛЖЕН БЫТЬ РИСУНОЧЕК

Таким образом, в задаче движения положение платформы в пространстве стола расматривается как ориентированная материальная точка.

Для определения положения платформы в пространстве в каждый момент времени рассматривается платформа, для которой известно количество тиков $p_1, p_2$, зафиксированная энкодерами каждого из колес, радиус колес $R$, расстояние между колесами $L$ и предыдущая тройка положения $(x_{previous}, y_{previous}, \alpha_{previous})$.

\section{Формулировка задачи движения}
Пусть известно первоначальное положение платформы $p_{start} = (x_{start}, y_{start}, \alpha_{start})$. Задан набор ориентированных точек $P = \{(x_i, y_i, \alpha_i)\}$, через которые необходимо проехать, допуская минимальные отклонения от предполагаемой траектории движения (прямой), не превышая заданные максимальные скорость и ускорение. В качестве критерия оптимизации алгоритмов принимается точность прибытия в каждую из точек множества $P$, минимизация перерегуляций (проезд дальше точки назначения) и минимазация затраченного на движение времени.

\chapter{ПИД-регулятор}

ПИД--регулятор --- пропорционально-интегрально-дифференциальный регулятор --- устройство в управляющем контуре с обратной связью. ПИД-регулятор формирует управляющий сигнал $u(t)$, составляющийся из трех слагаемых -- ошибка, интеграл ошибки, дифференциал ошибки -- каждое из которых взвешенно некоторым весом $K_P, K_I, K_D$, соотвественно. Под ошибкой (или невязкой) подразумевается разность $e = (x_0 - x) $, где $x$ --- некоторая величина, которую необходимо поддерживать заданным значением $x_0$ с помощью управляющего воздействия $u$.
\begin{equation}\label{pid_base}
u(t) = P + I + D = K_P e(t) + K_I \int\limits_0^t e(\tau) d\tau + K_D \frac{de}{dt}
\end{equation}

При решение практических задач функция ошибки обычно неизвестна, поэтому невозможно пользоваться аналитическими решениями интеграла и дифференциала, поэтому будем пользоваться численными методами интегрирования и дифференцирования.

Для численного интегрирования воспользуемся методом трапеции:

\begin{equation}\label{trapezoid}
\int\limits_a^b f(x) dx = \frac{f(a) + f(b)}{2}(b-a)
\end{equation}

Для численного дифференцирования будем пользоваться методом дифференцирования по трем узлам на правый край:

\begin{equation}\label{derivative}
f'(x) = \frac{3f(x) - 4f(x-h) + f(x-2h)}{2h}
\end{equation}

Теперь перепишем выражение \eqref{pid_base}, сделав замены в соответствии с уравнениями \eqref{trapezoid}, \eqref{derivative}:

\begin{equation}\label{pid_digit}
u(t_i) = K_P e(t_i) + K_I \frac{f(t_{i-1}) + f(t_i)}{2}(t_i - t_{i-1}) + K_D \frac{3e(t_i) - 4e(t_{i-1}) + e(t_{i-2})}{t_i - t_{i-2}}
\end{equation}
\chapter{Первый алгоритм (!)}

\section{Модель}

Попробуем в качестве решения искомой задачи использовать следующую модель --- будем использовать группу ПИД-регуляторов: корректор прямолийненого движения, корректор поворота на заданный абсолютный угол и корректор движения до заданной точки. Данный набор регуляторов позволяет задавать движение робота таким образом, что возможно проехать из любой произвольной точки в любую другую произвольную точку, если между ними существует непрерывный путь, допускающий движение платформы (платформа не задевает непреодолимых препятствий при движении).

\section{Корректор прямолинейного движения}

Для прямолинейного корректора задается базовая линейная скорость движения, с которой следует двигаться платформе.

\end{document}
