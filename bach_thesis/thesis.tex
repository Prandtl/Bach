\documentclass[12pt]{report}
\usepackage[russian]{babel}
\usepackage[utf8]{inputenc}
\usepackage{indentfirst}
\usepackage{amsmath,amssymb,amsfonts}
\usepackage{amsthm}
\usepackage{cite}
\usepackage[pdftex]{graphicx}

\textheight 24cm
\voffset -2cm
\textwidth 17cm
\hoffset -1.5cm
\topmargin 0mm

\renewcommand{\bottomfraction}{1.0}
  % Какая часть снизу листа может быть занята картинками
\renewcommand{\textfraction}{0.0}
  % Какая часть листа должна быть занята текстом

\newtheorem{theorem}{Теорема}
\newtheorem{definition}{Определение}
\newtheorem{ex}{Пример}
\newtheorem{tab}{Таблица}
\newtheorem{sat}{Утверждение}


\begin{document}
  \pagestyle{empty} % нумерация выкл.
  \begin{center}
    \small
    МИНИСТЕРСТВО ОБРАЗОВАНИЯ И НАУКИ РОССИЙСКОЙ ФЕДЕРАЦИИ\\
    Федеральное государственное автономное образовательное учреждение высшего образования\\
    УРАЛЬСКИЙ ФЕДЕРАЛЬНЫЙ УНИВЕРСИТЕТ \\
    имени первого Президента России Б.Н. Ельцина\\
    \vspace{1em}
    ИНСТИТУТ ЕСТЕСТВЕННЫХ НАУК И МАТЕМАТИКИ\\
    Кафедра математической экономики (TODO: Место выполнения ВКР и работы научного руководителя (консультанта) ВКР (либо кафедра, либо департамент))
  \end{center}

  \vspace{1em}

  \begin{center}
    \large
    АДАПТАЦИЯ АЛГОРИТМА ПАРАЛЛЕЛЬНОГО\\ СТОХАСТИЧЕСКОГО ГРАДИЕНТНОГО СПУСКА К ЗАДАЧАМ\\ МЕХАНИКИ СПЛОШНОЙ СРЕДЫ
    \vspace{1em}

    \normalsize
    Направление подготовки 09.03.03 <<Прикладная информатика>>\\
    \vspace{1em}
    Образовательная программа <<TODO: ???>>
    \end{center}
  \vspace{1em}

  \begin{tabular}[t]{@{}l}
    Допустить к защите:\\
    Директор департамента:\\ {TODO: title}М. О. Асанов\\
    \underline{\hspace{5cm}}\\
    \vspace{1em}\\
    Нормоконтроллер:\\ {TODO: find that person's name}\\
    \underline{\hspace{5cm}}
  \end{tabular}
  \hfill
  \begin{tabular}[t]{l@{}}
      Выпускная квалификационная\\работа бакалавра\\
      \textbf{Учанева}\\
      \textbf{Василия Вячеславовича}\\
      \underline{\hspace{5cm}}\\
      \vspace{1em}\\
      Научный руководитель:\\ к.ф.-м.н. В. С. Зверев\\
      \underline{\hspace{5cm}}\\
  \end{tabular}

  \vspace*{\fill}
  \begin{center}
    Екатеринбург 2017 г.
  \end{center}

  \newpage
  \pagestyle{plain} % нумерация вкл.
  \chapter*{Реферат}
  Учанев В. В. - -//- -: стр. - -,

  Ключевые слова: СТОХАСТИЧЕСКИЙ ГРАДИЕНТНЫЙ СПУСК, ПАРАЛЛЕЛИЗАЦИЯ.

  В данной работе рассматривается...{TODO:text}

  \newpage
  \tableofcontents

  \chapter{Введение}
  Оптимизация - поиск наилучшего элемента согласно некоторому критерию,
  например поиск максимального или минимального элемента. Для функции одной
  переменной существует аналитический метод поиска точек экстремума,
  но для функций двух и более переменных используются численные методы.

  Не смотря на то, что математические основы оптимизации были заложены
  в XVIII-XIX веке, широко использоваться они начали во второй половине XX века,
  в связи с появлением и распространением ЭВМ, поскольку использование
  методов оптимизации требует в большинстве случаев большой вычислительной работы.
  Поиск минимального/максимального значения используется во множестве областей:
  в экономике для оптимизации процессов и максимизации прибыли,
  в механике для проектирования и расчетов,
  в социальных науках для моделирования и оптимизации общественных процессов.

  За последние несколько лет был довольно большой скачок в развитии методов
  оптимизации, связанный c развитием машинного обучения. Методы оптимизации
  используются на этапе обучения: параметры модели подбираются так,
  чтобы уменьшить ошибку относительно тренировочного набора данных.

  Одним из наиболее используемых в машинном обучение алгоритмов является
  алгоритм градиентного спуска и его модификации. Причины этому в простоте его
  реализации и скорости работы, которая важна, так как во время обучения
  модель проходит через множество итераций и чем быстрее будет проходить
  каждая итерация, тем быстрее будет процесс поиска оптимальных параметров
  аппроксимирующей модели, то есть  обучение.

  \chapter{Постановка задачи}
\end{document}
